\chapter{Conclusiones} \label{chap:Conclusiones}
El área de \emph{mobile forensics}, en la cual se enmarca este trabajo, actualmente se encuentra en un momento interesante. Si bien la misma aún no cuenta con el nivel de estandarización de procedimientos forenses con el que se dispone en \emph{computer forensics}, en los últimos años han surgido una serie de iniciativas importantes que apuntan a la estandarización de varios de estos procedimientos. La tarea de extracción de datos de dispositivos móviles forma parte de dicho proceso y es en la que este trabajo estuvo enfocado con el fin de brindar solución a algunos de los desafíos que la misma presenta.

Una de las iniciativas interesantes que vimos es la llevada a cabo por MITRE, en colaboración con el \emph{Department of Homeland Security} (DHS) de los EEUU. Esta tiene como objetivo el desarrollo de varios lenguajes que buscan estandarizar y automatizar la forma en que se realiza el intercambio de la información de seguridad en diversas áreas. Uno de dichos lenguajes es CybOX, del cual hace uso la herramienta que desarrollamos para representar información sobre los datos extraídos.

El lenguaje CybOX fue la pieza clave que nos permitió cumplir con el objetivo de desarrollar una herramienta que produjera información en un formato fácilmente consumible por otras herramientas. Esto se debe a su adopción como estándar abierto y la madurez de las herramientas que brinda. A su vez, un aspecto particularmente destacable fue la gran disposición que mostraron de parte de MITRE para aclarar aspectos que no se encontraban documentados sobre las capacidades de extensibilidad del lenguaje y de la biblioteca \emph{python-cybox}. Esto fue importante ya que nos permitió validar que CybOX contaba con el poder de expresión y extensibilidad que buscábamos en el lenguaje a ser utilizado en la herramienta.

Recordemos que el objetivo principal que nos planteamos para la herramienta fue el de facilitar la extracción de datos de dispositivos móviles. Para esto, diseñamos la herramienta con un nivel de abstracción que permitiera a un investigador forense prescindir del conocimiento de cómo es implementada cada operación de extracción de datos que desea realizar. Para esto, concebimos cada operación como un procedimiento que extrae cierto tipo de datos, de una determinada fuente de datos y soportando un determinado conjunto de dispositivos móviles. Eso permitió que la herramienta pudiera contar con una interfaz de entrada la cual permite especificar los datos que el investigador desea obtener en términos que éste conoce.

El tercer y último objetivo que nos planteamos para la herramienta fue el de facilitar la extensión de la misma. Para ello, hicimos un gran énfasis en el diseño de la arquitectura. Las consideraciones realizadas para contar con una herramienta altamente extensible y con un mecanismo que permitiera importar fácilmente las extensiones desarrolladas fueron aspectos que resultaron esenciales. Esto se debe a que para poder adaptarnos a los rápidos cambios que presenta el ecosistema de dispositivos móviles y escalar para soportar una porción significativa de los mismos, es necesario contar con una forma de colaborar con otros. De esta manera, si podemos encapsular y compartir el conocimiento técnico que desarrollamos para que pueda ser reutilizado por otros, estaremos ampliando la capacidad con la que contamos para obtener datos.

El caso de estudio realizado permitió ilustrar las principales funcionalidades de la herramienta y la forma de desarrollar cada uno de los diversos tipos de extensión de la herramienta. De forma de realizar un caso representativo del uso real de la herramienta, tomamos las áreas de mensajería y redes sociales para desarrollar las extensiones. Además, obtuvimos un par de conjuntos de datos de prueba elaborados para este fin. Cabe mencionar que la obtención de dichos conjuntos de datos de prueba no fue una tarea sencilla dada la escasa disponibilidad de los mismos que hay públicamente en comparación a lo que podemos estar acostumbrados para otras áreas de seguridad informática. Creemos que esto se puede deber tanto al trabajo que implica sanitizar los datos como el hecho que el área es relevantemente nueva.

En cuanto al prototipo, logramos desarrollar una primera versión de la herramienta que permite mostrar las funcionalidades principales que nos planteamos obtener y desarrollamos un conjunto de extensiones para la misma en el área particular de datos de aplicaciones. El foco de su desarrollo estuvo puesto en contar con una arquitectura que luego permitiera continuar ampliando las funcionalidades de la herramienta en base a la visión que propusimos para la misma. En particular, vimos que hay varias ideas, vistas en el \autoref{chap:TrabajoFuturo}, que presentan oportunidades interesantes para continuar con el desarrollo de la herramienta, como son el desarrollo de extractors para otros dominios de datos y el soporte para otras plataformas móviles.

A su vez, es importante mencionar el proceso de desarrollo que tuvimos al implementar el prototipo. Una de las decisiones más importantes vista en retrospectiva fue la elección del lenguaje Python. Desde el comienzo fue un claro candidato dada la preferencia del mismo por parte de la comunidad forense open source y por el hecho de que la única biblioteca de alto nivel que provee CybOX es en el lenguaje Python. Para el desarrollo exitoso del prototipo además nos resultó muy importante la práctica de acompañar el desarrollo de cada módulo con el desarrollo de tests para el mismo. Esto nos permitió iterar rápidamente al detectar regresiones con facilidad. La versión final del prototipo cuenta con 55 tests unitarios cubriendo 88\% del total de líneas de código escrito.

A modo de cierre, el aporte que realiza el trabajo es bipartito. Por un lado, se realizó un estudio sobre el estado del arte de la extracción de datos de dispositivos móviles y las problemáticas que presenta dicha tarea. Por otro lado, en base a dicho estudio se desarrolló una herramienta para atacar varios de los problemas presentados. Como vimos, la herramienta fue desarrollada con un énfasis en el diseño de su arquitectura. De esta forma, otros interesados podrán extenderla a dominios de datos en el área de interés que deseen trabajar.
