\chapter{Trabajo a futuro}
En este capítulo plantearemos algunas ideas que fueron surgiendo en diversas etapas del desarrollo del trabajo. Creemos que las mismas presentan buenas oportunidades para mejorar el diseño de la herramienta y la implementación del prototipo desarrollado.

\section{Generalización de las condiciones impuestas por las operaciones}
Como vimos, una operación cuenta con información que condiciona los dispositivos que soportada. Esta información en el prototipo desarrollado corresponde únicamente a modelos de dispositivos y versiones de Android. Sería interesante desarrollar un mecanismo extensible que permita generalizar este concepto para permitir que las operaciones puedan imponer \emph{condiciones} sobre este tipo de información del dispositivo. 

De esta forma, los modelos y versiones de Android podrían pasar a ser condiciones particulares que impone una operación, en lugar de atributos fijos de las mismas. En caso que deseemos expresar que una operación soporta únicamente un cierto modelo de dispositivo, la operación simplemente tendría que imponer una condición de este tipo con el valor que representa al modelo de dispositivo correspondiente. Si en cambio la operación no desea restringir su uso según el modelo del dispositivo, la misma puede optar por no imponer condición alguna de este tipo permitiendo así su uso en cualquier modelo de dispositivo.

Veamos dos tipos de condiciones nuevas que consideramos que sería verdaderamente útil contar con ellas. Se tratan de si el dispositivo cuenta con \emph{acceso root} y de si el dispositivo tiene el \emph{modo debug} activado. En el prototipo desarrollado el usuario es el responsable de verificar si la operación que va a utilizar requiere alguno de estos dos y si es así asegurarse que el dispositivo cumple con los mismos. En caso de que una operación pueda imponer este tipo de condiciones, la herramienta sería capaz de verificar que las mismas se cumplen al compararlas con la información del dispositivo ingresada por el usuario sobre el dispositivo.

Para establecer la información del dispositivo contamos con el comando \texttt{set\_device\_info}. Actualmente el mismo solicita modelo del dispositivo y versión de Android que corre. Al agregar condiciones, como las mencionadas anteriormente, este comando debe pasar a solicitar esta información para que luego las operaciones puedan utilizar condiciones en función de la misma.

El comando \texttt{list\_operations}, que actualmente permite filtrar operaciones según sus campos (incluyendo modelo de dispositivo y versión de Android) pasaría a permitir filtrar operaciones también según los diversos tipos de condiciones. No casualmente la forma en que implementamos en el prototipo dicho comando hace que agregar esta funcionalidad sea muy sencillo. Los cambios que se requieren son agregar una nueva clase que herede de la clase abstract \emph{Filter} para cada tipo de condición (de forma similar a como hoy tenemos la clase \emph{AndroidVersionFilter} para filtrar por versión de Android) y luego ajustar el módulo de parsing de la entrada de línea de comando.

Por último, debemos notar que el comando \texttt{execute\_operations} continuará recibiendo como parámetro un objeto del tipo \emph{DeviceInfo} el cual contendrá la nueva información del dispositivo. Pensamos que sería razonable que esta información además de ser pasada al inspector como sucede actualmente, también aprovechemos a pasarla al extractor ya que podemos tener datos que le sean relevantes como si el dispositivo cuenta con el modo debug activado mencionado antes.


\section{Automatización de la obtención de la información del dispositivo}
El prototipo desarrollado cuenta con el comando \texttt{set\_device\_info} el cual es utilizado para ingresar manualmente la información del dispositivo, esto es, el modelo del mismo y la versión de Android que corre.

Dado que esta información resulta importante para la ejecución de las operaciones, la herramienta requiere que la misma sea especificada antes poder ejecutar cualquier operación. Por lo tanto, la obtención de esta información es una tarea indispensable que siempre debemos realizar. De esta forma, consideramos que automatizar esta tarea facilitaría el uso de la herramienta.

Para implementar esto, podría añadirse un nuevo comando para que desempeñe la tarea de obtención de la información de forma automática y luego establezca la información del dispositivo llamando internamente al comando \texttt{set\_device\_info}.

Cabe destacar que este comando no tiene intención alguna de sustituir al comando \texttt{set\_device\_info}, sino que su finalidad es la obtención y establecimiento de dicha información de forma automática. Además, consideramos que es esencial que siempre podamos contar con la posibilidad de establecer esta información de forma manual dado que el mecanismo que implemente la obtención de los datos de forma automática con seguridad no sea infalible.

\section{Caching de los datos extraídos de una fuente de datos}
Como vimos en la sección \ref{conceptos_fundamentales}, una operación extrae información de un tipo de dato de una determinada fuente de datos. Si en una sesión utilizamos varias operaciones que toman sus datos de la misma fuente de datos, la herramienta deberá realizar la misma extracción para cada una de ellas, debido a la forma en que se implementó el prototipo.

Este aspecto se puede solucionar si implementamos un mecanismo de caché para los datos de una misma fuente de datos que son obtenidos durante una sesión de ejecución de la herramienta.

Recordemos que las operaciones constan de dos componentes: extractor e inspector. Esto hace que el proceso de extracción de los datos sea totalmente independiente del de examinación de los mismos. Esta separación nos permite implementar el caché mencionado de forma transparente al usuario de la herramienta. La única diferencia que notará es que tras ejecutar una operación, las siguientes operaciones que utilicen la misma fuente de datos que esta última se realizarán más rápidamente.

De esta forma, si al ejecutar una operación ya se encuentra en caché los datos extraídos de la fuente de datos de donde pretende obtener datos la operación, la herramienta en vez de ejecutar el extractor correspondiente simplemente realiza una copia de los datos del caché.

Por otro lado, este mecanismo no sólo brinda una mejora de rendimiento, sino que también hace posible disminuir la cantidad de accesos que debemos realizar al dispositivo ayudando a prevenir modificaciones innecesarias que se puedan hacer al mismo de forma inadvertida.

Por último, el hecho de realizar una copia de los datos extraídos a partir del caché en vez de utilizar esos directamente asegura que los datos que examina el inspector de una operación no fueron modificados previamente por otro inspector.

\section{Mayores garantías en la preservación de los datos extraídos}
Hay algunos procedimientos que se podrían incorporar a la herramienta para que la misma brinde mayores garantías desde el punto de vista forense para la conservación de la integridad de los datos extraídos.

En primer lugar, podríamos llevar a cabo dos etapas de verificación:
\begin{itemize}
\item Verificar que la integridad de los datos extraídos fue conservada al realizar la extracción de los mismos del dispositivo. Para esto, debemos verificar que el hash de los archivos obtenidos del dispositivo coincide con los datos extraídos.
\item Verificar que los datos extraídos no son modificados al ser examinados por un inspector. Para esto, podríamos utilizar los hashes previamente obtenidos al realizar la extracción de los datos, computar el hash de los mismos tras finalizada la examinación y compararlos.
\end{itemize}

En segundo lugar, una posibilidad que tiene su trade-off interesante es la de limitar la libertad de los inspectors y obligarlos a utilizar un conjunto de primitivas (a ser desarrolladas). Entre las ventajas que brindaría esto se encuentra que podemos asegurar que el acceso a los archivos que realizan los inspectors sea de \emph{sólo lectura} únicamente y también poder llevar un log de todas las acciones realizadas por un inspector.

\section{Desarrollo de otros tipos de extractors}
Los extractors desarrollados en este trabajo todos utilizan el protocolo \emph{adb} para obtener los datos. El mismo nos resultó conveniente dado que permite interactuar de forma indiferente tanto con dispositivos virtuales (esto es, emuladores) como con dispositivos físicos (a través de USB).

Ahora bien, la única restricción con la cual fue diseñado el extractor fue que el mismo debe realizar una extracción lógica de la fuente de datos indicada y almacenar los archivos extraídos en el directorio que se le especifica. Por lo tanto, la elección de cómo extraer los datos queda en manos del desarrollador del extractor.

Dada esta libertad, una primera idea interesante a investigar sería la realización de extractors que obtengan datos del sistema operativo, en contraposición a los extractors que desarrollamos que obtienen datos de aplicaciones.

Otra idea que creemos que puede ser de sumo interés es el desarrollo de extractors que en vez de tomar datos a partir de un dispositivo conectado lo hacen a partir de una \emph{disk image}. Esto resultaría importante dado que permite seguir de más cerca los lineamientos que suelen establecer los diversos procesos forense, esto es, el contar con un respaldo de los datos extraídos y luego trabajar sobre una copia de los mismos en vez de trabajar sobre el dispositivo original. Lo más importante aún es que esto perfectamente puede ser implementado en un nuevo extractor con el prototipo actual, sin necesidad de modificar la herramienta.

\section{Añadir más datos del proceso al contenido CybOX generado}
Actualmente la herramienta añade algunos datos como son el nombre y versión de la misma (esto es, Android Inspector v1.0), un listado de archivos extraídos (con su respectiva información y hashes) y en cada objeto CybOX de los datos examinados referenciamos a los archivos de los cuales se obtuvo su información.

Dado que CybOX permite representar otros datos de interés como pueden ser información sobre los colaboradores que llevaron a cabo el proceso e información sobre el dispositivo del cual se realizó la extracción, puede ser interesante contemplar la posibilidad de agregar este tipo de información. 