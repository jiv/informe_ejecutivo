\chapter*{Resumen}
El área de \emph{mobile forensics} hoy en día tiene un papel importante en la informática forense debido al extenso uso que tienen los dispositivos móviles. Características como la fabricación de los mismos utilizando arquitecturas SoC (\emph{System on a Chip}) y la diversidad que presenta el ecosistema móvil, traen una serie de problemáticas desde el punto de vista forense.

Para realizar la extracción de datos de estos dispositivos, los investigadores forenses deben contar con una gran cantidad de herramientas. Cada una de estas, además de implicar un costo económico y de aprendizaje significativo, a menudo suele utilizar formatos propietarios para expresar sus resultados, lo cual dificulta la reutilización de los mismos.

El presente trabajo realiza un estudio del estado del arte de la extracción de datos de dispositivos móviles en el marco del proceso forense informático y profundiza sobre las diversas problemáticas que se encuentran al realizar esta tarea, particularmente en la plataforma Android.

Para solucionar varios aspectos de estos problemas, desarrollamos una herramienta que busca facilitar la extracción de datos de dispositivos móviles Android. La misma brinda un nivel de abstracción que permite al usuario especificar los datos que desea obtener sin necesidad de tener que saber cómo es realizada la extracción de los mismos. Además, ésta utiliza el lenguaje estándar y abierto CybOX (\emph{Cyber Observable eXpression}) para representar la información resultante de la extracción de datos.

Por último, la herramienta cuenta con un diseño que permite la extensibilidad de la misma con el fin de soportar tanto nuevas formas de extraer datos como nuevos dominios de datos. Esto brinda la posibilidad a un desarrollo colaborativo del conocimiento, de forma que el mismo pueda ser compartido para escalar ante la rápida evolución del ecosistema móvil.